\section{GUI Design}
\subsection{WPF Ressourcen}
Mit WPF Ressourcen kann man nützliche Objekte (Brushes, Styles, Templates etc.) zentral definieren und wiederverwenden.
\subsection{Resources}
\paragraph{Physikalische Ressourcen} In den File Properties kann man eine Datei als WPF-Ressource deklarieren welche dann in die Binärdatei einkompiliert werden. Zugegriffen kann dann mittels Resource-Key (relativer Pfadname).
\paragraph{Resource} Jedes beliebige Objekt in XAML kann als Ressource definiert werden. Es bird mit einem Key-Attribut und dem x-Namespace benannt.
\begin{lstlisting}[language=xml]
<Application.Resources>
    <SolidColorBrush x:Key ="MyButtonBackground" Color="#EEEEEE" />
</Application.Resources>
\end{lstlisting}
Unter diesem Key sind Ressourcen dann auch ansprechbar
\begin{lstlisting}[language=xml]
<Button Background="{StaticResource MyButtonBackground}" Content="Save" />
\end{lstlisting}
\paragraph{ResourceDictionary} Ist ein Behälter um Ressourcen zu speichern. Es indexiert nach dem Ressourcen-Namen (\verb+x:Key+) und kann in allen Elementen, welche von FrameworkElement ableiten, verwendet werden. Wenn auf eine Ressource zugegriffen werden möchte, wird \begin{enumerate*}[label=\itshape \arabic*\upshape.]\item Key im Element und allen Parent-Nodes gesucht (Logical Tree), \item dann wird der Key in \verb+Application.Resources+ gesucht, \item als letztes wird in System-Ressourcen gesucht.\end{enumerate*}. Die Reheinfolge im XAML-Code ist wichtig.
\paragraph{System Ressources} Auf Ressourcen in Namespace \verb+System.Windows+ kann man mittels statischer Properties zugegriffen werden. Dazu gehören: \verb+SystemColors+, \verb+SystemFonts+ und \verb+SystemParamters+.
\begin{lstlisting}[language=xml]
StaticResource="{StaticResource[name]}"
\end{lstlisting}
Die Statische Bindung macht CompileTime Check und findet Fehler früh.
\begin{lstlisting}[language=xml]
DynamicResource="{DynamicResource[name]}"
\end{lstlisting}
Anstelle von \verb+[name]+ steht der Key der Ressource
In der Dynamische Bindung wird ein RuntimeCheck gemacht und lässt dynamisch erzeugte und geladene Ressourcen zu. Beim Static Binding wird bei Objektkonstruktion ausgewertet, beim Dynamic Binding einmal pro Zugriff.
\paragraph{FindResource} Mit der Methode \verb+FrameworkElement.FindResources+  können Zugriffe auf XAML Resourcen gemacht werden.
\begin{lstlisting}
var okText = (string)FindResource("OkText");
var bgBrush = FindResource("DarkBrush") as Brush;
\end{lstlisting}

\subsubsection{Zugriff}

\begin{enumerate}
    \item Key wird im Element und in allen Parent-Nodes gesucht.
    \item Key wird in Application.Resources gesucht.
    \item Key wird in System-Ressourcen gesucht.
\end{enumerate}


\section{Data Trigger}
(in \code{Style.Triggers} eingebunden, wie vorher. Beispiel: Style ist für ListBoxItem, rote Schrift wenn nicht mehr an Lager (\code{InStock == 0})
\begin{lstlisting}
<DataTrigger Binding="{Binding InStock}" Value="0">
    <Setter Property="Foreground" Value="Red" />
</DataTrigger>
\end{lstlisting}

\section{Visual State Manager}

Ermöglicht innerhalb eines ControlTemplate-Elements unterschiedliche Darstellungen anhand eines Zustands.

Wurde erst später (nach Triggern) eingeführt => mit Silverlight => Nicht weiter Besprochen.

\section{Styles}

\begin{lstlisting}
<!-- App.xaml -->
<Application x:Class="ModernUi.App"
             xmlns="http://schemas.microsoft.com/winfx/2006/xaml/presentation"
             xmlns:x="http://schemas.microsoft.com/winfx/2006/xaml"
             xmlns:local="clr-namespace:ModernUi"
             StartupUri="MainWindow.xaml">
    <Application.Resources>
        <ResourceDictionary Source="style.xaml"></ResourceDictionary>
    </Application.Resources>
</Application>

<!-- Style.xaml -->
<Window.Resources>
    <!-- benannten Style als Basis benutzen -->
    <Style x:Key="DefaultButtonStyle" TargetType="Button">
      <Setter Property="BorderThickness" Value="1" />
      <Setter Property="BorderBrush" Value="Black" />
      <Setter Property="Background" Value="Transparent" />
      <Setter Property="Foreground" Value="Black" />
      <Setter Property="FontFamily" Value="Segoe UI" />
      <Setter Property="Margin" Value="2" />
      <Setter Property="Padding" Value="10 2 10 2" />
      <Setter Property="FontSize" Value="13" />
    </Style>
    <!-- Den obigen benannten Style erben und auf alle Buttons anwenden -->
    <Style TargetType="Button" BasedOn="{StaticResource DefaultButtonStyle}"> </Style>
    <!-- für den Help Button noch einen abgeleiteten Style definieren --> <Style x:Key="MyHelpButtonStyle" TargetType="Button"
           BasedOn="{StaticResource DefaultButtonStyle}">
      <Setter Property="Foreground" Value="Gray" />
      <Setter Property="BorderBrush" Value="Gray" />
    </Style>
</Window.Resources>

<!-- Window.xaml -->
<GradientStop Offset="0" Color="{StaticResource BaseColor}" />
\end{lstlisting}

\subsubsection{explizit}
\begin{lstlisting}
<Style x:Key="MyButtonStyle">
    <Setter Property="Button.Foreground" Value="#000000" />
    <Setter Property="Button.Background" Value="#FFFFFF" />
    <!-- auch komplexes möglich -->
    <Setter Property="Button.Background">
        <Setter.Value>
            <LinearGradientBrush StartPoint="0,0" EndPoint="0,1">
                <GradientStop Offset="0" Color="#dddddd"></GradientStop>
                <GradientStop Offset="0.5" Color="#F0F0F0"></GradientStop>
                <GradientStop Offset="1" Color="#dddddd"></GradientStop>
            </LinearGradientBrush>
        </Setter.Value>
    </Setter>
</Style>
\end{lstlisting}

Verwenden mit:\\\code{<Button Style="{StaticResource MyButtonStyle}" Content="Cancel" />}

\subsubsection{implizite Styles}
Man kann auch einen \code{TargetType="Button"} einfügen. Der Klassenname (\code{Button.Foreground}) kann dann weggelassen werden (\code{Foreground}).

Lässt man den Key weg, gilt der Style für alle Elemente des Controls

Man kann auch ableiten, mit \code{BasedOn={StaticResource MyButtonStyle}}.

Möchte man aber den Grundstil als Standard-Stil haben, muss man einen Sentinel einfügen, der den Key weglässt und nur BasedOn ist (der Grundstil muss einen Key haben, damit man davon ableiten kann)

\begin{lstlisting}[language=xml]
<Style TargetType="Button" x:Key="BaseButton"><!--Nutzung über Key-->
    <!-- verschiedene Setter -->
</Style>
<!-- Sentinel, gilt für alle Button -->
<Style TargetType="Button" BasedOn="{StaticResource BaseButton}" />
<!-- spezifischer Stil, Nutzung über Key -->
<Style TargetType="Button" x:Key="DisabledButton" 
    BasedOn="{StaticResource BaseButton}">
    <!-- verschiedene Setter -->
</Style>
\end{lstlisting}

Es gibt aber auch einen einfacheren Weg:
\begin{itemize}
    \item Default-Style belassen (TargetType ja, aber ohne ID)
    \item Sentinel-Style weglassen
    \item abgeleiteter Stil: \code{BasedOn="{StaticResource {x:Type Button}}"}
\end{itemize}

\begin{lstlisting}[language=xml]
<Style TargetType="Button"><!--Nutzung über Key-->
    <!-- verschiedene Setter -->
</Style>
<!-- spezifischer Stil, Nutzung über Key -->
<Style TargetType="Button" x:Key="DisabledButton" 
    BasedOn="{StaticResource {x:Type Button}}">
    <!-- verschiedene Setter -->
</Style>
\end{lstlisting}