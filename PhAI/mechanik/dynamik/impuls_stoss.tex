\subsubsection{Impuls und Stoss}
 \textbf{Impulserhaltungssatz} In einem abgeschlossenen System bleibt der Impuls erhalten. Wenn nur Kräfte zwischen zwei Körpern wirken (Kraft = Gegenkraft) bleibt der Impuls erhalten. Die Bewegung des Schwerpunktes ändert sich nicht durch die Kollision.
 
\textbf{Elastischer Stoss}  (z.B Billiardkugel) nach dem Stoss bleibt die kinetische Energie unverändert. Der Energieerhaltungssatz für die Bewegungsenergie sowie der Impulserhaltungssatz gilt. Es geht keine Energie verloren. Der Impuls vor dem Stoss = Impuls nach dem Stoss
\begin{itemize}
	\item bewegen sich zwei Objekte aufeinander zu, ist eine Geschwindigkeit vor dem Zusammenstoss negativ.
\end{itemize}

\textbf{Unelastischer Stoss} (z.B Autounfall).
nach dem Stoss ist die kinetische Energie kleiner. (wird in Wärme und Verformungsenergie umgewandelt) (nur der Impulserhaltungssatz gilt: $p_1 + p_2 = p_1^{'} + p_2^{'}$)


\begin{tabbing}
	\begin{tabu} to \linewidth {X l X l}
		\toprule
		Impuls & $\vec{p} = m \vec{v}$  &
		Kraftstoss & $\vec{I} = \Delta \vec{p} = \vec{F} \Delta t = m \Delta \vec{v}$ \\
		Elastischer Stoss (Obj 1) & $v_1^{'} = \frac{(m_1 - m_2) \cdot v_1 + 2 m_2 v_2}{m_1+m_2}$  &
		Elastischer Stoss (Obj 2) & $v_2^{'} = \frac{(m_2 - m_1) \cdot v_2 + 2 m_1 v_1}{m_2+m_1}$ \\
		Unelastischer Stoss & $v_1^{'} = v_2^{'} = \frac{m_1v_1 + m_2v_2}{m_1 + m_2}$ & Verformungsarbeit & $W = E_1 - E_2 = \frac{m_1m_2}{2(m_1+m_2)}(v_1-v_2)^2$ \\
	\end{tabu}
\end{tabbing}

\begin{tabbing}
	\begin{tabu} to \linewidth {l X l}
		Variable & Bedeutung & SI-Einheit \\
		\midrule
		$\vec{I}$ & Kraftstoss & $Ns = \frac{kg \cdot m}{s}$ \\
		$\vec{p}$ & Impulsänderung & $Ns = \frac{kg \cdot m}{s}$ \\
		$m$ & Masse des Körpers & $kg$ \\
		$\Delta v$ & Geschwindigkeitsänderung & $\frac{m}{s}$  \\
		$F$ & beschleunigte konstante Kraft & $N$ \\
		$\Delta t$ & Dauer der Krafteinwirkung & $s$ \\
		$v^{'}$ & Geschwindigkeit des Körpers nach dem Stoss & $\frac{m}{s}$\\
		$v$ & gemeinsame Geschwindigkeit beider Körper nach dem Stoss (unelastisch) & $\frac{m}{s}$ \\
		$W$ & Verformungsarbeit & $J$\\
		$E_1$ & Summe der Bewegungsenergie beider Körper vor dem Stoss & $J$\\
		$E_2$ & Summe der Bewegungsenergie beider Körper nach dem Stoss & $J$\\
		\bottomrule
	\end{tabu}
\end{tabbing}

\paragraph{Impuls anders erklärt}

$$p = m \cdot \Delta v = m \cdot a \cdot \Delta t = F \cdot \Delta t$$

Ein Stoss überträgt Impuls, wie Arbeit Energie überträgt. Wenn man eine Kraft über eine \textit{Distanz} anwendet, macht man \textbf{Arbeit} und erhöht die \textbf{Energie}. Wenn man eine Kraft über eine \textit{Zeitdauer} anwendet, macht man einen \textbf{Stoss} und erhöht den \textbf{Impuls}.

\paragraph{Schwerpunktgeschwindigkeit}
 Die Bewegung des Schwerpunktes ändert sich nicht durch die Kollision 
 $$  u = \frac{\sum p}{ \sum m} $$

Vor dem Stoss
$$ v_{1} = u + (v_{1} -u) = u + v_{1}^{rel} $$
$$ v_{2} = u + (v_{2} -u) = u + v_{2}^{rel} $$

Elastischer Stoss
$$ v_{1} = u - (v_{1} -u) = u - v_{1}^{rel} $$
$$ v_{2} = u - (v_{2} -u) = u - v_{2}^{rel} $$
$$ E_{kin} = \frac{m_{1}+m_{2}}{2}u^2+\frac{m_{1}}{2}(v_{1}-u)^2+\frac{m_{2}}{2}(v_{2}-u)^2 $$

Inelastischer Stoss
$$ v_{1}=v_{2}=u $$
$$ E_{kin} = \frac{m_{1}+m_{2}}{2}u^2$$

\paragraph{Vollkommen inelastischer Stoss}
Die Objekte bewegen sich nachher gemeinsam weiter. Impuls ist vor und nach dem Stoss gleich

Deformationsenergie
$$ Q = E_{kin1} + E_{kin2} - (E_{kin1}^{\prime}  + E_{kin2}^{\prime})  $$