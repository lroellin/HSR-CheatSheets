\section{Differntialgleichungen}

\subsection{Analytische Lösung}
\begin{enumerate}
    \item Homogenelösung $ y^{\prime \prime}(t) + by^{\prime}(t) + cy(t) = 0 $
    \item Partikulärlösung  $ y^{\prime \prime}(t) + by^{\prime}(t) + cy(t) = f(n) $
    \item Vollständige $ y(t) = y_{p}(t) + y_{h}(t) $
\end{enumerate}

\subsubsection{Vorgehen anhand Beispiel}
\begin{enumerate}
    \item Vorgegeben Differntialgleichung $ y^{\prime} + ay(t) = b $
    \item Homogenelösung suchen  $ y^{\prime} + ay(t) = 0 $
    \begin{enumerate}
        \item Ansatz wird vorgegeben $y_{h} = Ce^{kt}$
        \item Benötigte Ableitungen von Ansatz $y_{h}$ bilden 
        \item Ableitungen in Homogenelösung einsetzten und auflösen nach Variable
        \item $y_{h}$ = Ansatz und k durch Lösung von oben ersetzt
    \end{enumerate}
    \item Partikulärlösung suchen  $ y^{\prime} + ay(t) = b $
        \begin{enumerate}
        \item Ansatz wird vorgegeben $y_{p} =\frac{b}{a}$
        \item Benötigte Ableitungen von Ansatz $y_{p}$ bilden 
        \item Ableitungen in Partikulärlösung einsetzten und auflösen
        \item $y_{p}$ = Ansatz und Variable durch Lösung von oben ersetzt
    \end{enumerate}
    \item Vollständige $ y(t) = y_{p}(t) + y_{h}(t) $
\end{enumerate}

\subsection{Nummerische Lösung}
\subsubsection{1-Ordnung}
\textbf{Euler-Verfahren}
$$ f^{\prime}(t) \approx \frac{f(t+\Delta t) - f(t)}{\Delta t} $$
$$ f^{\prime}(t) \approx \frac{f_{n+1}-f_{n}}{\Delta t}$$

\textbf{Vorgehen}
\begin{enumerate}
    \item Differentialgleichung aufstellen / vorgegeben
    \item Ableitung durch Formel oben ersetzten
    \item Auflösen nach $f_{n+1}$
\end{enumerate}

\subsubsection{2-Ordnung}
Gegeben Differentialgleichung mit 2ter Ableitung (Nach 2te Ableitung aufgelöst)
$$ y^{\prime\prime} = v^{\prime} = g-\frac{k}{m} y(t) $$

Definiere Zwischenschritt
$$ y^{\prime} = v(t) $$

Darstellen als Vektoren/Matrizen-Gleichungssystem 
$$ U^{\prime}(t) = MU(t) + b $$

Vektor definieren (1 Reihe Stammfunktion, 2 Reihe 1 Ableitung)
$$ \vec{U}(t) = \begin{bmatrix} y(t) \\ v(t) \end{bmatrix} $$
 
Matrix definieren(1 Spalte Faktoren für  Stammfunktion, 2 Spalte Faktoren für 1 Ableitung, 1 Reihe Zwischenschritt, 2 Reihe Differentialgleichung)

$$ M = \begin{bmatrix} 0 & 1 \\ -\frac{k}{m} & 0 \end{bmatrix} $$

Konstantenvektor definieren (Konstanten einsetzten: 1 Reihe Hilfsfunktion, 2 Reihe Differentialgleichungen
 $$ \vec{b} = \begin{bmatrix} 0 \\ g \end{bmatrix} $$

Nun kann nach 1-Ordnung vorgegangen werden