\section{Sammelsurium}

\subsection{Umrechnungen/Einheiten}
\paragraph{Einheiten}
\begin{tabbing}
	\begin{tabu} to \linewidth {l l X}
		\toprule
		Vorsatz & Vorzeichen & Faktor \\
		\midrule
		Atto  & a & $10^{-18}$ \\
		Femto & f & $10^{-15}$ \\
		Pico & p & $10^{-12}$ \\
		Nano & n & $10^{-9}$ \\
		Mikro & $\mu$ & $10^{-6}$ \\
		Milli & m & $10^{-3}$ \\
		Zenti & c & $10^{-2}$ \\
		Dezi & d & $10^{-1}$ \\
		Hekto & h & $10^{2}$ \\
		Kilo & k & $10^{3}$ \\
		Mega & M & $10^{6}$ \\
		Giga & G & $10^{9}$ \\
		Tera & T & $10^{12}$\\
		Peta & P & $10^{15}$\\
		Exa & E & $10^{18}$ \\
		\bottomrule
	 \end{tabu}
\end{tabbing}


\newpage

\paragraph{Translation vs. Rotation} (TODO an einen Ort schieben, wo man es auch findet...)



\section{Gravitation}

\begin{tabbing}
	\begin{tabu} to \linewidth {l X}
		\toprule
		Name & Formel \\
		\midrule
		Gravitationskraft  & $ F_{G}(r) = G\frac{m_{1}m_{2}}{r^2} $ \\
		Gravitationspotential & $V_{G}(r) =  -G\frac{m_{1}m_{2}}{r} $ \\
		Bedingung geschlossene Bahn (Kin + pot kleiner 0) & $ \frac{m}{2} v^2 - G\frac{mM}{r}<0$ \\
		\bottomrule
	 \end{tabu}
\end{tabbing}


\section{Sonstiges}

Dichte $$\rho = \frac{m}{V}$$
Mitternachtsformel $$\frac{-b \pm \sqrt{b^{2}-4ac}}{2a}$$

\paragraph{RPM zu $\omega$}

RPM = Rounds per Minute, RPS = Rounds per Second

$$\omega = 2\pi * rps = 2\pi * \frac{rpm}{60s}$$

Zentrifuge mit $7.5*10^4 rpm$; $\omega = \frac{7.5*10^4 rpm * 2\pi}{60s} = 7853.98 \frac{rad}{s}$

\paragraph{Joule, Nm, Ws}

$$1 J = 1 Nm = 1 \frac{W}{s}$$

\paragraph{kW zu PS}

$$1kW = 1.36PS$$

$$116 PS = 85.29kW$$

\paragraph{kcal zu Joule}

$$1 kcal = 4184 J; 1 cal = 4.184J$$

\paragraph{Looping-Formel}

$$E_{pot0} = E_{kin} + E_{pot1}$$
$$E_{pot0} = m*g*h = m*g*(h_0 + h)$$
$$E_{pot1} = m*g*h = m*g*d = m*g*2r$$
$$E_{kin} = \frac{1}{2}*m*v^2$$
$$a_z = \frac{v^2}{r} \rightarrow a_z = g \rightarrow v^2 = g*r$$
$$E_{kin} = \frac{1}{2}*m*v^2 = \frac{1}{2}*m*g*r$$
$$E_{pot0} = E_{kin} + E_{pot1}$$
$$m*g*(h_0+h) = \frac{1}{2}*m*g*r + m*g*2r$$
$$h = \frac{5}{2}r-h_0$$

Falls Höhe $h$ vorgegeben:

$$F_N = m*a$$
$$a_z = \frac{v^2}{r} = g+a = g+\frac{F_N}{m}$$
$$\frac{v^2}{r} = g+\frac{F_N}{m}$$

\subsection{Looping}
Damit der Wagen den höchsten Punkt im Looping erreicht muss die kinetische Energie $E_{\text{kin}_0}$ des Wagens vor dem Looping gleich gross sein wie die Potentielle Energie $E_{\text{pot}_0}$, die es braucht um im Looping den höchsten Punkt zu erreichen:


\begin{equation*}
\begin{aligned}
h &= \text{Höhe des Loopings (Durchmesser)} \\
E_{\text{kin}_0} &= \frac{1}{2}  m v_0^2 \\
E_{\text{pot}_0} &= mgh \\
E_{\text{kin}_0} &= E_{\text{pot}_0} 
\Leftrightarrow \frac{1}{2}  m v_0^2 = mgh 
\Leftrightarrow v_0 = \sqrt{2gh} \\
\end{aligned}
\end{equation*}

Jetzt haben wir erst berechnet wie schnell der Wagen sein muss, damit er nach oben kommt. Oben ist die Geschwindigkeit jetzt aber $0 m/s$, er fällt also wieder Runter. Darum müssen wir jetzt noch ausrechnen wie schnell der Wagen sein muss damit er nicht runterfällt. Das ist der Fall wenn die Zentrifugalkraft $F_Z$ mindestens so hoch ist wie die Erdanziehungskraft $F_G$.


\begin{equation*}
\begin{aligned}
F_Z &= \frac{mv_1^2}{r} \\
F_G &= mg \\
F_Z &= F_G 
\Leftrightarrow \frac{mv^2}{r} = mg 
\Leftrightarrow v_1 = \sqrt{g * r} = \sqrt{g * \frac{h}{2}}
\end{aligned}
\end{equation*}

Die Anfangsgeschwindigkeit muss jetzt eine genug hohe kinetische Energie $E_{\text{kin}_0}$ haben, um nach oben zu kommen $E_{\text{pot}_0}$ und genug schnell zu sein um nicht Runterzufallen $E_{\text{kin}_1}$:
\begin{equation*}
\begin{aligned}
E_{\text{kin}_0} &= E_{\text{kin}_1} + E_{\text{pot}_0} \\
\Leftrightarrow \frac{1}{2} m v_0^2 &= \frac{1}{2} m v_1^2 + mgh &| \text{ $v_1$ einsetzen} \\
\Leftrightarrow \frac{1}{2} m v_0^2 &= \frac{1}{2} m (\sqrt{g\frac{h}{2}})^2 + mgh \\
\Leftrightarrow v_0 &= \sqrt{\frac{5}{2}gh}
\end{aligned}
\end{equation*}

\paragraph{Elastische Kollisionen (Vereinfachung)}

Siehe https://www.khanacademy.org/science/physics/linear-momentum/elastic-and-inelastic-collisions/v/deriving-the-shortcut-to-solve-elastic-collision-problems und https://www.khanacademy.org/science/physics/linear-momentum/elastic-and-inelastic-collisions/v/how-to-use-the-shortcut-for-solving-elastic-collisions

\paragraph{Kopieren}

Evtl. Ableitungs-/Integralregeln aus Analysis? (abtippen)